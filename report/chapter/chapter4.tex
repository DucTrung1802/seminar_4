\chapter{Thuật toán AprioriHybrid}

\section{Giới thiệu}

AprioriHybrid là sự kết hợp giữa hai thuật toán Apriori và AprioriTid nhằm tận dụng ưu điểm của cả hai. Các quan sát thực nghiệm cho thấy rằng Apriori hoạt động tốt trong các vòng lặp đầu tiên, trong khi AprioriTid trở nên hiệu quả hơn khi kích thước ứng viên tăng và bảng trung gian Ck\_bar trở nên nhỏ hơn so với cơ sở dữ liệu ban đầu. AprioriHybrid được thiết kế để tự động chuyển đổi từ Apriori sang AprioriTid tại thời điểm thích hợp, từ đó cải thiện hiệu năng tổng thể của quá trình khai phá tập mục thường xuyên.



\section{Ý tưởng của AprioriHybrid}

AprioriHybrid kết hợp hai thuật toán theo chiến lược sau:

\begin{itemize}
    \item Sử dụng Apriori trong các vòng lặp đầu tiên.
    \item Khi dự đoán rằng bảng Ck\_bar (nếu sử dụng AprioriTid) có thể nằm vừa trong bộ nhớ, thuật toán chuyển sang AprioriTid.
\end{itemize}

Chiến lược này tận dụng ưu điểm:

\begin{itemize}
    \item Apriori hiệu quả khi số lượng ứng viên còn lớn.
    \item AprioriTid hiệu quả khi ứng viên ít và có thể lưu trữ trong bộ nhớ.
\end{itemize}

\section{Quy trình tổng quát của AprioriHybrid}

Mặc dù không có giả mã chính thức tuyệt đối, quy trình kết hợp được mô tả như sau:

\begin{enumerate}
    \item Chạy Apriori từ mức 1 và tăng dần k.
    \item Ở mỗi mức k, ước lượng kích thước của bảng Ck\_bar nếu áp dụng AprioriTid.
    \item Nếu kích thước dự kiến của Ck\_bar nhỏ hơn hoặc bằng lượng bộ nhớ cho phép:
    \begin{itemize}
        \item Chuyển sang AprioriTid từ mức hiện tại.
        \item Tiếp tục chạy AprioriTid đến hết thuật toán.
    \end{itemize}
    \item Ngược lại, tiếp tục sử dụng Apriori.
\end{enumerate}

Không cần thiết phải chuyển ngược lại thành Apriori. Một khi đã chuyển sang AprioriTid, hiệu năng về sau luôn tốt hơn.

\section{Lợi ích của việc kết hợp}

AprioriHybrid thường vượt trội hơn Apriori trong hầu hết các trường hợp nhờ:

\begin{itemize}
    \item Tận dụng khả năng xử lý nhanh của Apriori ở mức thấp.
    \item Giảm mạnh chi phí quét dữ liệu khi chuyển sang AprioriTid.
    \item Khi kích thước Ck\_bar giảm dần theo k, thuật toán hưởng lợi từ hiệu năng tăng dần theo thời gian.
\end{itemize}

Theo các kết quả thực nghiệm, hiệu năng cải thiện nhiều nhất khi kích thước Ck\_bar giảm từ từ theo từng mức. Khi điều này xảy ra:

\begin{itemize}
    \item AprioriTid được kích hoạt sớm.
    \item Chi phí đọc đĩa giảm mạnh.
    \item Thời gian thực thi tổng thể giảm đáng kể.
\end{itemize}

Ngược lại, nếu kích thước Ck\_bar giảm đột ngột hoặc không giảm đều, thời điểm chuyển đổi có thể đến muộn hơn, nhưng hiệu năng tổng thể vẫn tốt hơn Apriori.

\section{Phân tích ưu điểm và hạn chế}

\subsection{Ưu điểm}

\begin{itemize}
    \item Tối ưu hóa tự động theo từng mức, phù hợp cho dữ liệu lớn.
    \item Tiết kiệm thời gian quét dữ liệu khi mức k cao.
    \item Giảm chi phí truy cập đĩa khi Ck\_bar vừa bộ nhớ.
    \item Thường nhanh hơn cả Apriori và AprioriTid trong thực tế.
\end{itemize}

\subsection{Hạn chế}

\begin{itemize}
    \item Cần ước lượng kích thước Ck\_bar, yêu cầu thêm logic trong cài đặt.
    \item Trong trường hợp dữ liệu rất nhỏ, lợi ích có thể không rõ ràng.
    \item Khi dữ liệu có nhiều giao dịch rất lớn, Ck\_bar ban đầu có thể quá lớn để việc chuyển đổi diễn ra sớm.
\end{itemize}

