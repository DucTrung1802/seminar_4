\chapter{Đánh giá hiệu năng và so sánh các thuật toán}
\section{Các yếu tố ảnh hưởng đến hiệu năng}

Hiệu năng của các thuật toán khai thác tập mục thường xuyên phụ thuộc mạnh vào đặc trưng dữ liệu và một số tham số điều khiển thuật toán. Dưới đây là những yếu tố quan trọng nhất:

1. Số lượng giao dịch (|D|)

Tập giao dịch càng lớn thì thời gian quét CSDL càng tăng. Đối với Apriori, số lần quét CSDL theo từng vòng lặp làm chi phí tăng tuyến tính theo |D|.

2. Số lượng mặt hàng (|I|)

Khi |I| lớn, không gian ứng viên có thể bùng nổ (combinatorial explosion), làm số lượng ứng viên tăng theo cấp số nhân. Điều này ảnh hưởng trực tiếp đến Apriori và AprioriTid.

3. Độ dài trung bình giao dịch (T)

Các giao dịch dài hơn làm tăng khả năng tạo ra nhiều tập k-mục tiềm năng $\rightarrow$ nhiều ứng viên hơn $\rightarrow$ thời gian kiểm tra hỗ trợ (support counting) lâu hơn.

4. Ngưỡng hỗ trợ tối thiểu (minsup)

minsup cao $\rightarrow$ ít tập thường xuyên $\rightarrow$ ít vòng lặp $\rightarrow$ chạy nhanh

minsup thấp $\rightarrow$ sinh ra rất nhiều ứng viên $\rightarrow$ thuật toán có thể chậm mạnh hoặc tốn nhiều bộ nhớ

5. Số lượng ứng viên sinh ra trong mỗi vòng lặp

Đây là yếu tố quyết định tốc độ: càng nhiều ứng viên, thuật toán càng chậm.

6. Chiến lược lưu trữ và truy xuất dữ liệu

Apriori dùng giao dịch gốc.
AprioriTid và Hybrid lưu bảng ID-list, giúp giảm kích thước dữ liệu về sau nhưng tốn bộ nhớ lúc đầu.




\section{Thiết kế thực nghiệm}
\subsection{Môi trường thực nghiệm}

(Điền thông tin hệ thống tại đây)
Ví dụ:

CPU: Intel i7 - 12700H

RAM: 16 GB

Hệ điều hành: Windows 11

Ngôn ngữ cài đặt thuật toán: Python

Phiên bản thư viện hỗ trợ (nếu có): efficient\_apriori




\subsection{Tập dữ liệu và cấu hình tham số}

Trong thực nghiệm, sử dụng 6 tập dữ liệu tổng hợp với các thông số được mô tả trong
Bảng~\ref{tab:dataset-config}.

\begin{table}[H]
    \centering
    \caption{Cấu hình các tập dữ liệu sử dụng trong thực nghiệm}
    \label{tab:dataset-config}

    \begin{tabular}{|c|c|c|c|p{8cm}|}
        \hline
        \textbf{Dataset} & \textbf{|D|} & \textbf{|I|} & \textbf{T} & \textbf{Minsup thử nghiệm}                       \\ \hline

        T5.I2.D100K      & 100000       & 2            & 5          & {0.25\%, 0.33\%, 0.5\%, 0.75\%, 1\%, 1.5\%, 2\%} \\ \hline
        T10.I2.D100K     & 100000       & 2            & 10         & {0.25\%, 0.33\%, 0.5\%, 0.75\%, 1\%, 1.5\%, 2\%} \\ \hline
        T10.I4.D100K     & 100000       & 4            & 10         & {0.25\%, 0.33\%, 0.5\%, 0.75\%, 1\%, 1.5\%, 2\%} \\ \hline
        T20.I2.D100K     & 100000       & 2            & 20         & {0.25\%, 0.33\%, 0.5\%, 0.75\%, 1\%, 1.5\%, 2\%} \\ \hline
        T20.I4.D100K     & 100000       & 4            & 20         & {0.25\%, 0.33\%, 0.5\%, 0.75\%, 1\%, 1.5\%, 2\%} \\ \hline
        T20.I6.D100K     & 100000       & 6            & 20         & {0.25\%, 0.33\%, 0.5\%, 0.75\%, 1\%, 1.5\%, 2\%} \\ \hline
    \end{tabular}
\end{table}




\subsection{Tiêu chí đánh giá}
\begin{itemize}
    \item Thời gian thực thi
    \item Số lần quét CSDL
    \item Số lượng và kích thước tập ứng viên
    \item Mức sử dụng bộ nhớ
\end{itemize}




\section{So sánh Apriori, AprioriTid và AprioriHybrid}
\subsection{Số lần quét CSDL}

Apriori: mỗi vòng lặp cần 1 lần quét lại toàn bộ CSDL $\rightarrow$ rất tốn chi phí khi số vòng lặp lớn.

AprioriTid: sau vòng đầu tiên, không cần quét CSDL nữa. Thuật toán chỉ làm việc trên bảng ID-list, giúp giảm chi phí I/O.

AprioriHybrid: kết hợp hai phương pháp.

Giai đoạn đầu dùng Apriori (khi ID-list còn lớn).

Khi kích thước ID-list nhỏ hơn CSDL gốc thì chuyển sang AprioriTid.

$\rightarrow$ Về lý thuyết: AprioriTid $\le$ Hybrid $\le$ Apriori (về số lần đọc CSDL).




\subsection{Chi phí sinh và kiểm tra ứng viên}

Apriori: tốn chi phí kiểm tra ứng viên trên toàn bộ giao dịch.

AprioriTid: sử dụng danh sách ID-list nên việc kiểm tra hỗ trợ nhanh hơn ở các vòng sau.

Hybrid: giảm chi phí ở những vòng sâu (vòng 3 trở đi).

Nhược điểm AprioriTid: ID-list ban đầu rất lớn $\rightarrow$ tốn bộ nhớ.




\subsection{Tốc độ thực thi tổng thể}

Tốc độ phụ thuộc vào:
\begin{itemize}
    \item Số vòng lặp
    \item Số lượng ứng viên sinh ra
    \item Chi phí đọc/ghi dữ liệu
    \item Kết luận thường gặp:
    \item minsup cao $\rightarrow$ cả ba thuật toán đều nhanh
    \item minsup thấp $\rightarrow$ Apriori chậm nhất, AprioriTid hoặc Hybrid tốt hơn tùy dataset
\end{itemize}




\subsection{Mức sử dụng bộ nhớ}

Apriori: tiết kiệm bộ nhớ, chỉ giữ dữ liệu gốc + ứng viên hiện tại.

AprioriTid: chiếm bộ nhớ lớn vì phải lưu bảng ID-list cho từng ứng viên.

Hybrid: dùng nhiều bộ nhớ ở giai đoạn chuyển tiếp nhưng nhìn chung tiết kiệm hơn AprioriTid.




\subsection{Ảnh hưởng của minsup (thấp / cao)}

minsup cao
$\rightarrow$ ít ứng viên $\rightarrow$ cả 3 thuật toán chạy nhanh $\rightarrow$ Apriori có thể ngang bằng hoặc nhanh hơn do không tốn ID-list.

minsup thấp
$\rightarrow$ nhiều ứng viên $\rightarrow$ Apriori chậm nhất vì phải quét lại CSDL.
$\rightarrow$ AprioriTid và Hybrid vượt trội hơn, đặc biệt khi số vòng lặp lớn.




\subsection{Kịch bản phù hợp cho từng thuật toán}

Apriori: dataset nhỏ hoặc minsup cao.

AprioriTid: dataset lớn, minsup thấp, số vòng lặp nhiều.

Hybrid: trường hợp tổng quát – thường tốt nhất nếu không biết trước đặc trưng dữ liệu.





\section{Kết quả thực nghiệm}

Trong phần này, chúng tôi trình bày các kết quả thu được từ quá trình đánh giá hiệu năng của ba thuật toán
Apriori, AprioriTid và AprioriHybrid. Kết quả bao gồm các bảng tổng hợp số liệu và các biểu đồ trực quan
giúp so sánh tốc độ thực thi, số lượng ứng viên và mức sử dụng bộ nhớ.

% =======================
% BẢNG 1
% =======================

\begin{figure}[H]
    \centering
    \includegraphics[width=0.95\textwidth]{img/bảng 1.pdf}
    \caption{Bảng số lượng ứng viên theo mỗi vòng lặp của ba thuật toán}
    \label{fig:bang1}
\end{figure}

% =======================
% BẢNG 2
% =======================

\begin{figure}[H]
    \centering
    \includegraphics[width=0.95\textwidth]{img/bảng 2.pdf}
    \caption{Bảng thời gian thực thi của Apriori, AprioriTid và AprioriHybrid}
    \label{fig:bang2}
\end{figure}

% =======================
% HÌNH 1
% =======================

\begin{figure}[H]
    \centering
    \includegraphics[width=0.9\textwidth]{img/hình 1.pdf}
    \caption{Biểu đồ số thời gian chạy cho Dataset T5.I2.D100K}
    \label{fig:hinh1}
\end{figure}

% =======================
% HÌNH 2
% =======================

\begin{figure}[H]
    \centering
    \includegraphics[width=0.9\textwidth]{img/hình 2.pdf}
    \caption{Biểu đồ số thời gian chạy cho Dataset T10.I2.D100K}
    \label{fig:hinh2}
\end{figure}

% =======================
% HÌNH 3
% =======================
\begin{figure}[H]
    \centering
    \includegraphics[width=0.9\textwidth]{img/hình 3.pdf}
    \caption{Biểu đồ số thời gian chạy cho Dataset T10.I4.D100K}
    \label{fig:hinh3}
\end{figure}

% =======================
% HÌNH 4
% =======================

\begin{figure}[H]
    \centering
    \includegraphics[width=0.9\textwidth]{img/hình 4.pdf}
    \caption{Biểu đồ số thời gian chạy cho Dataset T20.I2.D100K}
    \label{fig:hinh4}
\end{figure}

% =======================
% HÌNH 5
% =======================

\begin{figure}[H]
    \centering
    \includegraphics[width=0.9\textwidth]{img/hình 5.pdf}
    \caption{Biểu đồ số thời gian chạy cho Dataset T20.I4.D100K}
    \label{fig:hinh5}
\end{figure}

% =======================
% HÌNH 6
% =======================

\begin{figure}[H]
    \centering
    \includegraphics[width=0.9\textwidth]{img/hình 6.pdf}
    \caption{Biểu đồ số thời gian chạy cho Dataset T20.I6.D100K}
    \label{fig:hinh6}
\end{figure}




\section{Nhận xét và thảo luận}
\subsection{Ưu và nhược điểm từng thuật toán}

Apriori:

Ưu: đơn giản, dễ cài đặt, ít tốn RAM

Nhược: nhiều lần quét CSDL, không hiệu quả khi minsup thấp

AprioriTid:

Ưu: không quét lại CSDL sau vòng đầu

Nhược: danh sách ID-list lớn, dễ tràn bộ nhớ

Hybrid:

Ưu: cân bằng giữa hai thuật toán, thường nhanh nhất

Nhược: logic phức tạp hơn, cần điều kiện chuyển chế độ



\subsection{Giải thích sự khác biệt giữa ba phương pháp}

Lý do chủ yếu: cách lưu trữ và cách tính hỗ trợ khác nhau.

\begin{itemize}
    \item Apriori dùng dữ liệu gốc
    \item AprioriTid dùng ID-list
    \item Hybrid chuyển đổi giữa hai giai đoạn
\end{itemize}




\subsection{Các trường hợp sử dụng thích hợp}

Dựa trên các đặc điểm về cấu trúc dữ liệu, chi phí xử lý và mức độ tiêu tốn tài nguyên, mỗi thuật toán trong ba thuật toán Apriori, AprioriTid và AprioriHybrid phù hợp với các tình huống ứng dụng khác nhau. Phần này trình bày các kịch bản sử dụng điển hình cho từng thuật toán.

\subsubsection*{Thuật toán Apriori}

Thuật toán Apriori phù hợp trong các trường hợp:
\begin{itemize}
    \item Tập dữ liệu có kích thước nhỏ hoặc trung bình, với số vòng lặp không quá lớn.
    \item Ngưỡng hỗ trợ tối thiểu (\textit{minsup}) tương đối cao, giúp giảm số lượng tập ứng viên.
    \item Hệ thống ưu tiên tính đơn giản, dễ triển khai và dễ kiểm soát.
    \item Các bài toán phân tích giao dịch ở quy mô vừa, chẳng hạn như
          phân tích giỏ hàng trong các cửa hàng bán lẻ nhỏ hoặc tập dữ liệu hành vi người dùng có quy mô hạn chế
          (<ví dụ: dữ liệu giỏ hàng từ một chuỗi cửa hàng nhỏ, dữ liệu clickstream của một website quy mô nhỏ>).
\end{itemize}

\subsubsection*{Thuật toán AprioriTid}

AprioriTid thích hợp với các tình huống:
\begin{itemize}
    \item Ngưỡng hỗ trợ thấp, khiến số lượng ứng viên tăng mạnh qua từng vòng lặp.
    \item Tập dữ liệu có số lượng giao dịch lớn nhưng độ dài mỗi giao dịch không quá lớn.
    \item Hệ thống có đủ bộ nhớ để lưu bảng ID-list trong quá trình thực thi.
    \item Các ứng dụng khai thác luật kết hợp yêu cầu xử lý khối lượng lớn dữ liệu, ví dụ:
          phân tích giỏ hàng trong các hệ thống bán lẻ quy mô lớn hoặc phân tích log hành vi người dùng trong các nền tảng thương mại điện tử
          (<ví dụ: dataset giao dịch của siêu thị lớn, dữ liệu clickstream của sàn thương mại điện tử>).
\end{itemize}

\subsubsection*{Thuật toán AprioriHybrid}

AprioriHybrid kết hợp ưu điểm của hai thuật toán trên và phù hợp với phần lớn các tình huống thực tế:
\begin{itemize}
    \item Các tập dữ liệu lớn và đa dạng, với đặc trưng không ổn định hoặc khó dự đoán trước.
    \item Các trường hợp yêu cầu cân bằng giữa tốc độ thực thi và mức sử dụng bộ nhớ.
    \item Khi hệ thống cần xử lý nhiều loại dữ liệu khác nhau, tránh phụ thuộc vào một thuật toán cố định.
    \item Các ứng dụng phân tích nâng cao trong thương mại điện tử, marketing và tài chính, chẳng hạn như
          phân tích thị trường, xây dựng hệ thống gợi ý sản phẩm hoặc khai thác mẫu chi tiêu của khách hàng
          (<ví dụ: dữ liệu giao dịch ngân hàng, dữ liệu hành vi người dùng của nền tảng streaming>).
\end{itemize}




\subsection{Hạn chế và hướng cải thiện}

Mặc dù các thuật toán Apriori, AprioriTid và AprioriHybrid được sử dụng rộng rãi trong khai thác luật kết hợp, chúng vẫn tồn tại một số hạn chế nhất định. Phần này trình bày các hạn chế chính và gợi ý một số hướng cải thiện tiềm năng trong tương lai.

\subsubsection*{Hạn chế}

\begin{itemize}
    \item \textbf{Chi phí sinh ứng viên lớn:} 
    Tất cả các thuật toán dựa trên Apriori đều phụ thuộc vào chiến lược sinh--kiểm tra (\textit{generate-and-test}), dẫn đến số lượng ứng viên tăng theo cấp số nhân khi \textit{minsup} thấp hoặc khi tập mặt hàng lớn.

    \item \textbf{Nhiều lần quét cơ sở dữ liệu:} 
    Apriori yêu cầu quét toàn bộ cơ sở dữ liệu nhiều lần, gây tiêu tốn thời gian I/O đáng kể đối với các tập dữ liệu lớn.

    \item \textbf{Tốn bộ nhớ (AprioriTid và Hybrid):} 
    Việc duy trì các bảng ID-list trong AprioriTid và AprioriHybrid làm tăng đáng kể mức sử dụng bộ nhớ, đặc biệt ở các vòng lặp đầu.

    \item \textbf{Không phù hợp với dữ liệu rất lớn:}
    Các thuật toán Apriori truyền thống không hiệu quả trong môi trường dữ liệu lớn (\textit{big data}) do chi phí xử lý cao và không tận dụng được khả năng tính toán song song hiện đại.

    \item \textbf{Không tối ưu cho dữ liệu có cấu trúc phức tạp:}
    Các thuật toán này chủ yếu phù hợp với dữ liệu dạng giao dịch; chúng không hoạt động tốt với dữ liệu có cấu trúc phức tạp như chuỗi thời gian, dữ liệu phân cấp hoặc dữ liệu đồ thị.
\end{itemize}





\subsubsection*{Hướng cải thiện}

\begin{itemize}
    \item \textbf{Sử dụng các thuật toán thay thế không sinh ứng viên:} 
    Các thuật toán như FP-Growth hoặc ECLAT giúp loại bỏ chi phí sinh ứng viên và có khả năng mở rộng tốt hơn đối với tập dữ liệu lớn.

    \item \textbf{Tối ưu hóa truy xuất dữ liệu:}
    Áp dụng các kỹ thuật như nén giao dịch, đánh chỉ mục hoặc sử dụng cấu trúc dữ liệu hiệu quả hơn để giảm chi phí quét CSDL.

    \item \textbf{Khai thác tính song song và phân tán:} 
    Việc triển khai trên các nền tảng như Hadoop MapReduce hoặc Spark cho phép mở rộng thuật toán tới quy mô dữ liệu rất lớn.

    \item \textbf{Điều chỉnh động ngưỡng \textit{minsup}:} 
    Một số nghiên cứu đề xuất thích ứng ngưỡng hỗ trợ theo cấu trúc dữ liệu giúp giảm số lượng ứng viên và tăng hiệu quả khai thác.

    \item \textbf{Ứng dụng học máy để giảm không gian tìm kiếm:} 
    Các mô hình học máy hoặc học sâu có thể hỗ trợ dự đoán các tập mục tiềm năng, qua đó giảm số lượng ứng viên phải xét trong từng vòng lặp.

    \item \textbf{Phát triển phiên bản tối ưu theo đặc trưng dữ liệu:}
    Tuỳ chỉnh thuật toán cho từng loại dữ liệu cụ thể 
    (<ví dụ: dữ liệu giao dịch thời gian thực, dữ liệu tuần tự, dữ liệu đồ thị>) 
    giúp tăng hiệu quả rõ rệt.
\end{itemize}
